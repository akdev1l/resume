%!TEX TS-program = xelatex
\documentclass[]{friggeri-cv}
\usepackage{afterpage}
\usepackage{hyperref}
\usepackage{color}
\usepackage{xcolor}
\usepackage{progressbar}
\hypersetup{
    pdftitle={Edgard(Alex) Arvelaez - Resume},
    pdfauthor={Edgard(Alex) Arvelaez},
    pdfsubject={Resume},
    pdfkeywords={resume,office,communications,skill,edgard,arvelaez,devops},
    colorlinks=false,       % no lik border color
   allbordercolors=white    % white border color for all
}
% \addbibresource{bibliography.bib}
\RequirePackage{xcolor}
\definecolor{pblue}{HTML}{0395DE}

\begin{document}
\header{Alex}{ Arvelaez}
      {}
      
% Fake text to add separator      
\fcolorbox{white}{gray}{\parbox{\dimexpr\textwidth-2\fboxsep-2\fboxrule}{%
.....
}}

% In the aside, each new line forces a line break
\begin{aside}
  \section{Address}
    Toronto, ON, Canada
    ~
  \section{Telephone}
    +1 (647) - 606 2058
    ~
  \section{Mail}
    \href{mailto:alex@akdev.xyz}{\textbf{alex@}\\akdev.xyz}
    ~
  \section{Technical Skills}
    \begin{tabular}{p{1.1cm} p{2.5cm}}
      \progressbar[width=1.1cm,filledcolor=green]{1} & {Linux Admin} \\
      \progressbar[width=1.1cm,filledcolor=green]{1} & {Shell} \\
      \progressbar[width=1.1cm,filledcolor=green]{0.9} & {Ansible} \\
      \progressbar[width=1.1cm,filledcolor=green]{0.85} & {Jenkins} \\
      \progressbar[width=1.1cm,filledcolor=green]{0.85} & {Python} \\
      \progressbar[width=1.1cm,filledcolor=green]{0.8} & {Documentation} \\
    \end{tabular}
    ~
  \section{Personal Skills}
    \begin{tabular}{p{1.2cm} p{2.0cm}}
      \progressbar[width=1.1cm,filledcolor=blue]{1.0} & {Accountability} \\
      \progressbar[width=1.1cm,filledcolor=blue]{1.0} & {Collaboration} \\
      \progressbar[width=1.1cm,filledcolor=blue]{1.0} & {Organization} \\
    \end{tabular}
    ~
  \section{Projects}
    \bullet\hspace{0.2cm}Personal Intranet
    \bullet\hspace{0.2cm}Pi3 Cluster
    \bullet\hspace{0.2cm}Python Bots
    \bullet\hspace{0.2cm}Resume in LaTeX
    ~
  \section{Technologies}
    \bullet\hspace{0.2cm}New Relic
    \bullet\hspace{0.2cm}ELK
    \bullet\hspace{0.2cm}Docker
    ~
  \section{Github}
    \href{https://github.com/e4lejandr0}{https://github.com/e4lejandr0}
    ~
\end{aside}

\section{Experience}
\begin{entrylist}
  \entry
    {01/18 - 04/18}
    {DevOps Engineer}
    {Fundserv}
    {Impletemented a fully-automated pipeline to provision VMs on VMware ESXi using Jenkins and Ansible.\\
    Developed automated build, package and deployment procedure which was implemented across multiple applications with a centralized ci infrastructure.\\
    Created standardized systemd services for easy startup and shutdown of applications.\\
    Developed customized ansible inventory plugin to get inventory information via the vSphere Automation SDK.\\}
  \entry
    {05/18 - 08/19}
    {Software Developer(Co-Op)}
    {Fundserv}
    {Developing new or making changes to existing software programs, web or mobile applications.\\
    Assisting in automation of new or existing processes.\\
    Designing and implementing scalable, reliable and easily maintainable solutions.\\}
  \entry
    {01/18 - 04/18}
    {Peer Tutor}
    {Seneca College}
    {Provided one-on-one academic support in specific subject areas.\\
    Facilitated independent learning by helping tutees discover their own answers and insights.\\
    Provided academic assistance by personalizing instruction and reviewing course material.\\}
  \entry
    {05/17 - 12/17}
    {Unix Automation Developer(Co-Op)}
    {Fundserv}
    {Maintained an internal website for the System Operations team.\\
    Developed automation of alerts and health checks as well as report generating for the internal website.\\
    Developed a new automated process to startup the company's applications.\\}
\end{entrylist}

\section{Education}
\begin{entrylist}
  \entry
    {2019}
    {Red Hat Certified Systems Administrator}
    {Red Hat}
    {Completed two Red Hat courses and was able to successfully obtain the RHCSA certification.}
  \entry
    {2019}
    {Red Hat Certified Ansible Specialist}
    {Red Hat}
    {Self-taugh Ansible and was able to obtain the specialist-level certification on it.\\
    \emph{}\\}
  \entry
    {2016 - 2018}
    {Computer Programming Diploma}
    {Seneca College}
    {Strong focus on hands-on experience with industry standard technologies .\\
    Main subjects: Object-Oriented Software Development Using C++, Database Design II and SQL Using Oracle, Java for C++ Programmers.\\
    {GPA: \emph{3.8}\\}
    \emph{}\\}
\end{entrylist}
\end{document}
