%!TEX TS-program = xelatex
\documentclass[]{friggeri-cv}
\usepackage{afterpage}
\usepackage{hyperref}
\usepackage{color}
\usepackage{xcolor}
\usepackage{progressbar}
% remove link decorations: https://tex.stackexchange.com/questions/823/remove-ugly-borders-around-clickable-cross-references-and-hyperlinks
\hypersetup{
    pdftitle={Alex Díaz (akdev/akdev1l) - Resume},
    pdfauthor={akdev}
    pdfsubject={Alex Díaz - Resume}
    pdfkeywords={resume,programming,java,c++,c,python,rust,alex,devops,sysadmin,developer,software,edgard,arvelaez,aws,lambda,linux,kernel,ublue,os},
    pdfborder=0 0 0,
    colorlinks=false,       % no lik border color
    allbordercolors=white    % white border color for all
}
\RequirePackage{xcolor}
\definecolor{pblue}{HTML}{0395DE}

\begin{document}
\header{Alex}{ Díaz}
    {}

% Fake text to add separator      
\fcolorbox{white}{gray}{\parbox{\dimexpr\textwidth-2\fboxsep-2\fboxrule}{%
.....
}}

% In the aside, each new line forces a line break
\begin{aside}
  \section{Location}
    Toronto, ON, Canada
    ~
  \section{Github}
    \href{https://github.com/akdev1l}{https://github.com/akdev1l}
    ~
  \section{Mail}
    \href{mailto:alex@akdev.xyz}{\textbf{alex@}akdev.xyz}
    ~
  \section{Programming Languages}
    \begin{tabular}{p{1.1cm} p{2.5cm}}
      \progressbar[width=1.1cm,filledcolor=green]{1.0} & {Java} \\
      \progressbar[width=1.1cm,filledcolor=green]{1.0} & {C} \\
      \progressbar[width=1.1cm,filledcolor=green]{1.0} & {C++} \\
      \progressbar[width=1.1cm,filledcolor=green]{1.0} & {Python} \\
      \progressbar[width=1.1cm,filledcolor=green]{1.0} & {BASH} \\
      \progressbar[width=1.1cm,filledcolor=green]{0.9} & {TypeScript} \\
      \progressbar[width=1.1cm,filledcolor=green]{0.9} & {JavaScript} \\
      \progressbar[width=1.1cm,filledcolor=green]{0.8} & {Rust} \\
    \end{tabular}
    ~
  \section{Personal Skills}
    \begin{tabular}{p{1.2cm} p{2.0cm}}
      \progressbar[width=1.1cm,filledcolor=blue]{1.0} & {Ownership} \\
      \progressbar[width=1.1cm,filledcolor=blue]{1.0} & {Collaboration} \\
      \progressbar[width=1.1cm,filledcolor=blue]{1.0} & {Organization} \\
    \end{tabular}
    ~
  \section{Projects}
    \item[\rightarrow]\href{https://github.com/akdev1l/ostree-images}{OSTree Images}
    \item[\rightarrow]\href{https://github.com/ublue-os/isogenerator}{UBlue isogenerator}
    \item[\rightarrow]\href{https://github.com/akdev1l/container-images}{Container Images}
    ~
  \section{Technologies}
    \item[\rightarrow]{AWS CDK}
    \item[\rightarrow]{CloudFormation}
    \item[\rightarrow]{Docker/Podman}
    \item[\rightarrow]{Ansible}
    \item[\rightarrow]{libvirt/qemu/kvm}
    \item[\rightarrow]{Linux}
    ~
\end{aside}

\section{Experience}
\begin{entrylist}
  \entry
    {December 2023 - Present}
    {Sr. Systems Development Engineer - L6}
    {Amazon}
    {\begin{itemize}
        \item Supervise development of multiple software projects.
        \item Set timelines, prioritize tasks and communicate progress to management and stakeholders.
        \item Establish sprint planning, code review and testing standards across multiple projects.
        \item Provide guidance and mentoring to other engineers to maximize team velocity.
        \item Review and provide feedback on architectural designs for new services.
    \end{itemize}}
  \entry
    {April 2023 - November 2023}
    {Systems Development Engineer - L5}
    {Amazon}
    {\begin{itemize}
        \item Built process automation platform which resulted in over \$5MM annualized savings.
        \item Created tasks and architectural designs while leading Junior engineers on implementation.
        \item Took the lead on technical decisions across the team's projects due to experience and highly technical background.
        \item Participated in code review process and deliver feedback effectively to increase team productivity.
        \item Implement technically involved parts of ongoing projects (authentication, database access, performance-related functionality).
    \end{itemize}}
  \entry
    {January 2023 - April 2023}
    {DevOps Engineer - L5}
    {Amazon}
    {\begin{itemize}
        \item Created Web-based tool for recovery during emergencies. It automated a clunky and error-prone manual process which resulted in savings of ~1500 engineer hours per year. (AWS API Gateway, AWS Lambda, DynamoDB, SQS)
        \item Designed the architecture of the back-end application on AWS Lambda to lower ongoing maintenance costs. (Java, Dagger, AWS Lambda, Smithy)
        \item Designed and implemented the front-end website using React, Redux and TypeScript following best practices to achieve optimal performance and responsiveness in mobile and under poor connectivity. (TypeScript, React, Redux, API Gateway, AWS Lambda)
        \item Designed data processing platform to analyze and detect stuck shipments within the Amazon Fulfillment Network. Expected to save over 4000 hours of engineer time per year. (AWS Glue, Athena, ECS)
    \end{itemize}}
  \entry
    {June 2020 - December 2022}
    {DevOps Engineer - L4}
    {Amazon}
    {\begin{itemize}
        \item Support multiple teams on high severity and low severity incidents, decreasing resolution time.
        \item Developed a pipeline scanning tool to analyze full CD/CI best practices on over 1,000 pipelines.
        \item Supported the use of the pipeline scanning tool leading the improvement of the FullCD posture of 30 pipelines.
        \item Improved an internal tool for automatic dashboard generation taking previously defined alert thresholds into account.
    \end{itemize}}
  \entry
    {May 2017 - May 2020}
    {Systems Engineer}
    {Fundserv}
    {\begin{itemize}
        \item Implemented a fully-automated pipeline to provision VMs on VMware ESXi\\using Jenkins and Ansible.
        \item Developed customized ansible inventory plugin to get inventory information via the vSphere Automation SDK.
        \item Developed a three-tiered Angular/Springboot Java application for accounting.
        \item Developed a new service management facility built on ksh88.
    \end{itemize}}
\end{entrylist}

\section{Education}
\begin{entrylist}
  \entry
    {2021}
    {AWS Cloud Practitioner}
    {Amazon Web Services}
    {Obtained the Cloud Practitioner certification as an opening in the AWS Skill Tree.}
  \entry
    {2019}
    {Red Hat Certified Systems Administrator}
    {Red Hat}
    {Obtained the RHCSA certification after attending Red Hat training sessions.}
  \entry
    {2019}
    {Red Hat Certified Ansible Specialist}
    {Red Hat}
    {Self-taught Ansible and was able to obtain the specialist-level certification on it.}
  \entry
    {2018}
    {Computer Programming Diploma}
    {Seneca College}
    {Strong focus on hands-on experience with industry standard technologies.}
\end{entrylist}
\end{document}
