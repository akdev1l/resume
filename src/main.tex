%!TEX TS-program = xelatex
\documentclass[]{friggeri-cv}
\usepackage{afterpage}
\usepackage{hyperref}
\usepackage{color}
\usepackage{xcolor}
\usepackage{progressbar}
\hypersetup{
    pdftitle={Edgard(Alex) Arvelaez - Resume},
    pdfauthor={Edgard(Alex) Arvelaez},
    pdfsubject={Resume},
    pdfkeywords={resume,office,communications,skill,edgard,arvelaez,devops},
    colorlinks=false,       % no lik border color
   allbordercolors=white    % white border color for all
}
% \addbibresource{bibliography.bib}
\RequirePackage{xcolor}
\definecolor{pblue}{HTML}{0395DE}

\begin{document}
\header{Alex}{ Arvelaez}
      {}
      
% Fake text to add separator      
\fcolorbox{white}{gray}{\parbox{\dimexpr\textwidth-2\fboxsep-2\fboxrule}{%
.....
}}

% In the aside, each new line forces a line break
\begin{aside}
  \section{Address}
    Toronto, ON, Canada
    ~
  \section{Github}
    https://github.com/akdev1l
    ~
  \section{Mail}
    \href{mailto:alex@akdev.xyz}{\textbf{alex@}\\akdev.xyz}
    ~
  \section{Technical Skills}
    \begin{tabular}{p{1.1cm} p{2.5cm}}
      \progressbar[width=1.1cm,filledcolor=green]{1} & {Linux Admin} \\
      \progressbar[width=1.1cm,filledcolor=green]{1} & {AWS} \\
      \progressbar[width=1.1cm,filledcolor=green]{0.9} & {Ansible} \\
      \progressbar[width=1.1cm,filledcolor=green]{0.85} & {Java} \\
      \progressbar[width=1.1cm,filledcolor=green]{0.85} & {Python} \\
      \progressbar[width=1.1cm,filledcolor=green]{0.8} & {Documentation} \\
    \end{tabular}
    ~
  \section{Personal Skills}
    \begin{tabular}{p{1.2cm} p{2.0cm}}
      \progressbar[width=1.1cm,filledcolor=blue]{1.0} & {Ownership} \\
      \progressbar[width=1.1cm,filledcolor=blue]{1.0} & {Collaboration} \\
      \progressbar[width=1.1cm,filledcolor=blue]{1.0} & {Organization} \\
    \end{tabular}
    ~
  \section{Projects}
    \bullet\hspace{0.2cm}Personal Intranet
    \bullet\hspace{0.2cm}Pi3 Cluster
    \bullet\hspace{0.2cm}Python Bots
    \bullet\hspace{0.2cm}Resume in LaTeX
    ~
  \section{Technologies}
    \bullet\hspace{0.2cm}AWS CDK
    \bullet\hspace{0.2cm}Java
    \bullet\hspace{0.2cm}Python
    ~
\end{aside}

\section{Experience}
\begin{entrylist}
  \entry
    {06/20 - Now}
    {DevOps Engineer}
    {Amazon}
    {Support multiple teams on high severity and low severity incidents decreasing resolution time. \\
    Simplified and automated manual onboarding procedures to increase system stability and minimize ongoing manual effort.
    Adjusted alarms to minimize false alerts to avoid alarm fatigue and decrease manual effort in ticket triaging.
    Documented frequent steps for common incidents allowing for faster and more consistent incident resolution.}
  \entry
    {10/18 - 05/2020}
    {Systems Engineer}
    {Fundserv}
    {Impletemented a fully-automated pipeline to provision VMs on VMware ESXi using Jenkins and Ansible.\\
    Developed automated build, package and deployment procedure which was implemented across multiple applications with a centralized ci infrastructure.\\
    Created standardized systemd services for easy startup and shutdown of applications.\\
    Developed customized ansible inventory plugin to get inventory information via the vSphere Automation SDK.\\}
  \entry
    {05/18 - 09/18}
    {Software Developer(Co-Op)}
    {Fundserv}
    {Developed an Angular/Springboot-based Java application from scratch. Provided a REST API served to \\
    a web application powered by Angular 2. \\
    Assisted in automation of new and existing processes.\\
    Designed and implemented scalable, reliable and easily maintainable solutions.\\}
  \entry
    {05/17 - 12/17}
    {Unix Automation Developer(Co-Op)}
    {Fundserv}
    {Maintained an internal website for the Systems Operations team.\\
    Developed automation of alerts and health checks as well as report generating for the internal website.\\
    Developed a new automated process to startup the company's applications.\\}
\end{entrylist}

\section{Education}
\begin{entrylist}
  \entry
    {2019}
    {Red Hat Certified Systems Administrator}
    {Red Hat}
    {Completed two Red Hat courses and was able to successfully obtain the RHCSA certification.}
  \entry
    {2019}
    {Red Hat Certified Ansible Specialist}
    {Red Hat}
    {Self-taugh Ansible and was able to obtain the specialist-level certification on it.\\
    \emph{}\\}
  \entry
    {2016 - 2018}
    {Computer Programming Diploma}
    {Seneca College}
    {Strong focus on hands-on experience with industry standard technologies .\\
    {GPA: \emph{3.8}\\}
    \\}
\end{entrylist}
\let\clearpage\relax
\end{document}
