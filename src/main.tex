%!TEX TS-program = xelatex
\documentclass[]{friggeri-cv}
\usepackage{afterpage}
\usepackage{hyperref}
\usepackage{color}
\usepackage{xcolor}
\usepackage{progressbar}
% remove link decorations: https://tex.stackexchange.com/questions/823/remove-ugly-borders-around-clickable-cross-references-and-hyperlinks
\hypersetup{
    pdftitle={Edgard(Alex) Arvelaez - Resume},
    pdfauthor={Edgard(Alex) Arvelaez},
    pdfsubject={Resume},
    pdfkeywords={resume,office,communications,skill,edgard,arvelaez,devops},
    pdfborder=0 0 0,
    colorlinks=false,       % no lik border color
    allbordercolors=white    % white border color for all
}
\RequirePackage{xcolor}
\definecolor{pblue}{HTML}{0395DE}

\begin{document}
\header{Alex}{ Diaz}
    {}

% Fake text to add separator      
\fcolorbox{white}{gray}{\parbox{\dimexpr\textwidth-2\fboxsep-2\fboxrule}{%
.....
}}

% In the aside, each new line forces a line break
\begin{aside}
  \section{Location}
    Toronto, ON, Canada
    ~
  \section{Github}
    \href{https://github.com/akdev1l}{https://github.com/akdev1l}
    ~
  \section{Mail}
    \href{mailto:alex@akdev.xyz}{\textbf{alex@}akdev.xyz}
    ~
  \section{Technical Skills}
    \begin{tabular}{p{1.1cm} p{2.5cm}}
      \progressbar[width=1.1cm,filledcolor=green]{1} & {Linux Admin} \\
      \progressbar[width=1.1cm,filledcolor=green]{1} & {AWS} \\
      \progressbar[width=1.1cm,filledcolor=green]{0.9} & {Ansible} \\
      \progressbar[width=1.1cm,filledcolor=green]{0.85} & {Java} \\
      \progressbar[width=1.1cm,filledcolor=green]{0.85} & {Python} \\
      \progressbar[width=1.1cm,filledcolor=green]{0.8} & {Documentation} \\
    \end{tabular}
    ~
  \section{Personal Skills}
    \begin{tabular}{p{1.2cm} p{2.0cm}}
      \progressbar[width=1.1cm,filledcolor=blue]{1.0} & {Ownership} \\
      \progressbar[width=1.1cm,filledcolor=blue]{1.0} & {Collaboration} \\
      \progressbar[width=1.1cm,filledcolor=blue]{1.0} & {Organization} \\
    \end{tabular}
    ~
  \section{Projects}
    \bullet\hspace{0.2cm}Personal Intranet
    \bullet\hspace{0.2cm}\href{https://github.com/dokutan/mouse_m908/pull/12}{mouse\_m908}
    \bullet\hspace{0.2cm}\href{https://gitlab.freedesktop.org/NetworkManager/NetworkManager/-/merge_requests/679}{nmcli ovs integration}
    \bullet\hspace{0.2cm}\href{https://bugs.almalinux.org/view.php?id=151}{AlmaLinux Ticket Triaging}
    ~
  \section{Technologies}
    \bullet\hspace{0.2cm}AWS CDK
    \bullet\hspace{0.2cm}CloudFormation
    \bullet\hspace{0.2cm}Jenkins
    \bullet\hspace{0.2cm}Ansible
    \bullet\hspace{0.2cm}libvirt/qemu/kvm
    \bullet\hspace{0.2cm}Linux
    ~
\end{aside}

\section{Experience}
\begin{entrylist}
  \entry
    {June, 2020 - Present}
    {DevOps Engineer}
    {Amazon}
    {\begin{itemize}
        \item Support multiple teams on high severity and low severity incidents, decreasing resolution time.
        \item Developed a pipeline scanning tool to analyze full CD/CI best practices on over 1,000 pipelines.
        \item Adjusted alarms to minimize false alerts to avoid alarm fatigue and decrease manual effort in ticket triaging.
        \item Migrated an internal website from AL1 (based on RHEL5) to AL2 by updating the build system, updating all dependencies and fixing broken functionality.
        \item Improved an internal tool for automatic dashboard generation taking previously defined alert thresholds into account.
    \end{itemize}}
  \entry
    {May, 2017 - May, 2020}
    {Systems Engineer}
    {Fundserv}
    {\begin{itemize}
        \item Implemented a fully-automated pipeline to provision VMs on VMware ESXi\\using Jenkins and Ansible.
        \item Developed automated build, package and deployment procedure which was implemented across multiple applications with a centralized CI infrastructure.
        \item Created standardized systemd services for easy startup and shutdown of\\applications.
        \item Developed customized ansible inventory plugin to get inventory information via the vSphere Automation SDK.
        \item Developed a three-tiered Angular/Springboot Java application for accounting.
        \item Developed and maintained an internal monitoring website for the Systems\\Operations team.
        \item Developed a new service management facility built on ksh88.
    \end{itemize}}
\end{entrylist}

\section{Education}
\begin{entrylist}
  \entry
    {2021}
    {AWS Cloud Practitioner}
    {Amazon Web Services}
    {Obtained the Cloud Practitioner certitication as an opening in the AWS Skill Tree.}
  \entry
    {2019}
    {Red Hat Certified Systems Administrator}
    {Red Hat}
    {Obtained the RHCSA certification after attending Red Hat training sessions.}
  \entry
    {2019}
    {Red Hat Certified Ansible Specialist}
    {Red Hat}
    {Self-taught Ansible and was able to obtain the specialist-level certification on it.}
  \entry
    {2018}
    {Computer Programming Diploma}
    {Seneca College}
    {Strong focus on hands-on experience with industry standard technologies.}
\end{entrylist}
\end{document}
